\documentclass[twocolumn]{article}
\usepackage[letterpaper, margin=0.5in]{geometry}
\usepackage{amsmath}
\usepackage{amssymb}
\usepackage{mathtools}
\DeclarePairedDelimiter{\ceil}{\lceil}{\rceil}
\begin{document}
\textbf{\underline{CS 22: Midterm 2 Study Sheet}} \\

\textbf{Lecture 12 (2/23): Fundamental Theorem of Arithmetic}
\begin{itemize}
    \item Every $x \in \mathbb{Z}$, $x>1$ can be written as a product of prime numbers
    \item Proof: COME BACK TO
\end{itemize}

\textbf{Lecture 12 (2/23): Divisibility}
\begin{itemize}
    \item $a|b$ over $\mathbb{Z}$ if $\exists c \in \mathbb{Z}$ s.t. $ac = b$ 
    \item $a|b$ is read as a divides b
\end{itemize}

\textbf{Lecture 12 (2/23): Primes}
\begin{itemize}
    \item Theorem: there are infinitely many primes
    \item Proof: COME BACK TO
    \item Theorem: Let p(n) = number of primes $<$ n, then $\textbf{lim}_{n\rightarrow \inf}p(n) = \frac{n}{\ln{n}}$
    \item Goldback Conjecture: Every even integer $>2$ is the sum of two primes
\end{itemize}

\textbf{Lecture 13 (2/26): Division Algorithm}
\begin{itemize}
    \item given $a \in \mathbb{Z}$ and $d \in \mathbb{Z}^+$, $\exists$ unique $q, r \in \mathbb{Z}$ s.t. $a=dq+r$ where $0 \leq r < d$
\end{itemize}

\textbf{Lecture 13 (2/26): Euclidean Algorithm}
\begin{itemize}
    \item the Euclidean algorithm calculates the greatest common divisor of a and b
    \item gcd is a loop invariant: gcd(a,b) = gcd(b, rem(a, b)) where $a, b \in \mathbb{Z}^+$
    \item input: (a,b) where $a, b \in \mathbb{Z}^+$
    $$a = bq_0 + r_0, 0 \leq r_0 < b$$
    $$b = r_0q_1+r_1, 0 \leq r_1 < r_0$$
    $$r_0 = r_1q_2 + r_2, 0 \leq r_2 < r_1$$
    \item let s be the smallest integer such that $r_s = 0$
    \begin{itemize}
        \item if s=0 then gcd(a, b) = b
        \item if $s \neq 0$ then gcd(a, b) = $r_{s-1}$
    \end{itemize}
    \item example: gcd(480, 105) = 15
    $$480 = 105*4+60$$
    $$105 = 60*1 + 45$$
    $$60 = 45*1 + 15$$
    $$45 = 15*2$$
\end{itemize}

\textbf{Lecture 13 (2/26): Euclidean Algorithm Proof of correctness}
\begin{itemize}
    \item involves loop invariant
\end{itemize}

\textbf{Lecture 13 (2/26): Linear Combinations}
\begin{itemize}
    \item Theorem: $\forall a, b \in \mathbb{Z}^+, \exists u, v \in \mathbb{Z}$ s.t. gcd(a, b) = a*u+b*v
    \item can find u and v through backtrack euclidean algorithm
    \item gcd(5, 17) = 1, 1 = 5*7 - 17*2
    \item basically express numbers in last equation that is equal to 1 to be in terms of previous steps until represent 1 fully in terms of a and b
    \item Proof: COME BACK TO
\end{itemize}

\textbf{Lecture 14 (2/28): Modular Arithmetic}
\begin{itemize}
    \item let $a, b \in \mathbb{Z}$ and $m \in \mathbb{Z}^+$
    \item define R on $\mathbb{Z}$ where aRb if $m|(a-b)$
    \item aRb can be rewritten as $a \equiv b$ (mod m), an equivalence relation
    \item mod function: xMODm gives the remainder of x divided by m
\end{itemize}

\textbf{Lecture 14 (3/2): Modular Properties}
\begin{itemize}
    \item mod properties: if $a \equiv b$ (mod m) and $c \equiv d$ (mod m), then:
    \begin{itemize}
        \item $a+c \equiv b+d$ (mod m)
        \item $ac \equiv bd$ (mod m)
        \item $a^n \equiv b^n$ (mod m)
        \item $a+f \equiv b+f$ (mod m)
        \item $af \equiv bf$ (mod m)
        \item $a-f \equiv b-f$ (mod m)
    \end{itemize}
    \item can't always cancel
\end{itemize}

\textbf{Lecture 14 (3/2): Solution Exists to Modular Congruence}
\begin{itemize}
    \item $ax \equiv b$ (mod m) has a solution iff gcd(a, m)$|$b
    \item Proof: COME BACK TO
    \item Special case: $ax\equiv 1$ (mod m) has a solution iff gcd(a, m) = 1
\end{itemize}

\textbf{Lecture 14 (3/2): Multiplicative Inverse}
\begin{itemize}
    \item let gcd(a, m) = 1
    \item if $ax \equiv ya$ (mod m), then $x \equiv y$ (mod m); can be shown by multiplying both sides by the multiplicative inverse of a
    \item gcd(a, m) = 1, $\exists z$ s.t. $az \equiv 1$ (mod m)
    \item z is the multiplicative inverse of a
\end{itemize}

\textbf{Lecture 14 (3/2): Fermat's Little Theorem}
\begin{itemize}
    \item Let p be prime, if gcd(a, p) = 1, then $$a^{p-1} \equiv 1 \text{(mod p)}$$
    \item given that $ax \equiv 1$ ( mod p), $x = a^{p-2}$
    \item Proof: COME BACK TO
\end{itemize}

\textbf{Lecture 15 (3/5): Euler's Totient (Phi) Function}
\begin{itemize}
    \item COME BACK TO
\end{itemize}

\textbf{Lecture 15 (3/5): Phi function}
\begin{itemize}
    \item for p, q prime $p \neq q$, $\phi(p*q) = (p-1)*(q-1)$
    \item Proof: COME BACK TO
\end{itemize}

\textbf{Lecture 15 (3/5): Euler-Fermat}
\begin{itemize}
    \item if for a, m, gcd(a, m) = 1, then $a^{\phi(m)}\equiv 1$ (mod m).
\end{itemize}

\textbf{Lecture 16 (3/7): Public Key Cryptosystem}
\begin{itemize}
    \item Steps:
    \begin{enumerate}
        \item p, q: choose 2 primes p and q; n = pq
        \item $\phi(n):$ calculate $\phi(n)=(p-1)(q-1)$
        \item k: choose a k such that $1<k<\phi(n)$ and $gcd(k, \phi(n))=1$
        \item d: find a d such that $kd \equiv 1 (\text{mod}\phi(n))$
        \item public: n (the modulus) and k (the encryption exponent)
        \item private: d
    \end{enumerate}
    \item Encryption: to encrypt message m, compute $r = m^k \text{ (mod n)}$
    \item Decryption: to decrypt message r calculate $x = r^d \text{ (mod n)}$
    \item Proof of RSA: COME BACK TO
\end{itemize}

\textbf{Lecture 17 (3/9): Notation}
\begin{itemize}
    \item $\lnot$ means not
    \item $\wedge$ means and
    \item $\vee$ means or
    \item $\oplus$ means xor (or but not both of)
    \item $\implies$ means implies
    \item $\iff$ means if and only if (p implies q and q implies p)
    \item
        \begin{tabular}{ |c|c|c|c|c| }
            \hline
            p & q & $p \implies q$ & $p \iff q$ \\
            \hline
            T & T & T & T\\ 
            T & F & F & F\\ 
            F & T & T & F\\ 
            F & F & T & T\\ 
            \hline
        \end{tabular}
    \item $q \implies p$ is the converse of $p \implies q$
    \item $\lnot p \implies \lnot q$ is the inverse of $p \implies q$
    \item $\lnot q \implies \lnot p$ is the contrapositive (logically equivalent) of $p \implies q$
    \item tautology: proposition that is always true $p \vee \lnot p$
    \item contradiction: proposition that is always false $p \wedge \lnot p$
\end{itemize}

\textbf{Lecture 18 (3/12): Proof by Contradiction}
\begin{itemize}
    \item Want to show that predicate p is true
    \item find contradiction q such that $\lnot p \implies q$ is true
    \item the above implies that $\lnot p$ is false, so p must be true
    \item If p then q; assume that q is not true and show that if q is not true then p is not true (contrapositive)
\end{itemize}

\textbf{Lecture 18 (3/12): Normal Forms}
\begin{itemize}
    \item disjunctive form: OR of ANDS
    \item conjunctive form: AND of ORS
    \item SAT: question of whether or not there exists an assignment to the variables which satisfy the expression (boolean satisfiability problem)
    \item 3-SAT: special case of SAT problem where the boolean expression should be divided into clauses (expressions in parentheses) where each clause contains 3 literals
\end{itemize}

\textbf{Lecture 18 (3/12): Boolean Algebra}
\begin{itemize}
    \item same as propositional logic except instead of T and F, use 1's and 0's
    \item boolean function: $f: \{ 0, 1\}^n \rightarrow \{ 0, 1\}$
    \item operations:
    \begin{itemize}
        \item addition is OR
        \item multiplication is AND
        \item inverse is NOT
    \end{itemize}
    \item Identities:
    \begin{itemize}
        \item x+x = x
        \item x*x = x
        \item x+1 = 1
        \item x+0 = x
    \end{itemize}
\end{itemize}

\textbf{Lecture 19 (3/14): Combinatorial Circuits}
\begin{itemize}
    \item AND, OR, NOT, XOR gates
    \item combinatorial circuits: combining multiple gates
\end{itemize}

\textbf{Lecture 19 (3/14): Binary Representation}
\begin{itemize}
    \item $\forall n \in \mathbb{Z}_{\geq 0}\exists$ an expression of the form $n=a_k*2^k+a_{k-1}**2^{k-1}+...+a_1*2^1+a_0*2^0$ where $a_i \in \{0, 1\}\forall i$
    \item example: $1011 = 1*2^3+0*2^2+1*2^1+1*2^0$
\end{itemize}

\textbf{Lecture 19 (3/14): Half Adder}
\begin{itemize}
    \item adding two bits requires sum and carry bits
    \item 
        \begin{tabular}{ |c|c|c|c|c| }
            \hline
            x & y & S & C \\
            \hline
            1 & 1 & 0 & 1\\ 
            1 & 0 & 1 & 0\\ 
            0 & 1 & 1 & 0\\ 
            0 & 0 & 0 & 0\\ 
            \hline
        \end{tabular}
    \item S = $x \oplus y$; $C = x\wedge y$
\end{itemize}

\textbf{Lecture 20 (3/16): Full Adder}
\begin{itemize}
    \item full adder: basically just adding an extra thing to add (to account for previous column carrying in numbers)
    \item composed of two half adders
    \item
        \begin{tabular}{ |c|c|c|c|c|c| }
            \hline
            x & y & $C_{in}$ & S & $C_{out}$ \\
            \hline
            1 & 1 & 1 & 1 & 1\\ 
            1 & 1 & 0 & 0 & 1\\ 
            1 & 0 & 1 & 0 & 1\\ 
            1 & 0 & 0 & 1 & 0\\ 
            0 & 1 & 1 & 0 & 1\\ 
            0 & 1 & 0 & 1 & 0\\ 
            0 & 0 & 1 & 1 & 0\\ 
            0 & 0 & 0 & 0 & 0\\ 
            \hline
        \end{tabular}
\end{itemize}

\textbf{Lecture 20 (3/16): Ripple Adder}
\begin{itemize}
    \item full adder only adds one column in the binary addition of two numbers
    \item in order to add multi-bit binary numbers, use ripple adder
    \item ripple adder is a group of full adders ending with a half adder (half adder because first addition has no carry in)
    \item ripple adder uses previous adder's carry bit as it's input
\end{itemize}

\textbf{Lecture 20 (3/16): Feedback Circuit}
\begin{itemize}
    \item takes output of circuit and runs it back as input
    \item theta in class notes diagram represents output
\end{itemize}

\textbf{Lecture 21 (3/19): SR Latch}
\begin{itemize}
    \item allows us to either store
and hold Q or change its value by flipping an input
\end{itemize}

\textbf{Lecture 21 (3/19): Clock and D Latch}
\begin{itemize}
    \item clock signal indicates when the latch should open, read the value of D, and store it
    \item D = data
    \item C= 0 means “latch closed”
    \item C = 1 means value of Q becomes value of D
\end{itemize}

\textbf{Lecture 22 (3/21): Flip Flop Circuit}

\textbf{Lecture 22 (3/21): 2 Bit Counter}

\textbf{Lecture 23 (3/23): Produce Rule}
\begin{itemize}
    \item Product rule: Let S be a set of length-k sequences; $|S|=n_1*n_2*...*n_k$ COME BACK TO
    \item Proof of product rule (easy induction): COME BACK TO
\end{itemize}

\textbf{Lecture 23 (3/23): Counting functions}
\begin{itemize}
    \item $|X|=n, |Y|=m$
    \item number of functions $f:X \rightarrow Y$ is $m^n$
    \item number of functions $f:X\rightarrow Y$ that are injective (one-to-one) is $m(m-1)...(m-n-1)$; application of product rule to number of elements in co domain
    \item hard to count number of surjective (onto) functions
    \item number of bijective functions is the same as the number of injective functions
    \item Proposition: if $|X|=|Y|$ and $f:X\rightarrow Y$ is injective, then f is surjective
    \item basically if number of bijections is the number of injections if the size of the sets is the same (IS THIS IFF?)
\end{itemize}

\textbf{Lecture 23 (3/23): Factorial}
\begin{itemize}
    \item definition of n factorial:
    for $n \i \mathbb{Z}$, $n\geq 1$:
    $n! = n(n-1)...(1)$
    \item define $0!=1$ 
\end{itemize}

\textbf{Lecture 23 (3/23): Permutations}
\begin{itemize}
    \item a permutation of a set S is a sequence that contains every element of S exactly once (if n elements, n! possible permutations)
    \item given $f: X \rightarrow X$, if f is a bijection, then it is called a permutation; (if $|X|=n$, then the number of permutations of X is $n!$)
\end{itemize}

\textbf{Lecture 23 (3/23): Combinations}
\begin{itemize}
    \item for $n \geq k \geq 0 \in \mathbb{Z}$:
     $${n \choose k} = \frac{n!}{(n-k)!(k)!}$$
    \item ${n \choose k}$ is the number of ways to select k objects from an n element set; also equal to number of subsets of size k of an n-element set
    \item number of subsets of n-element sets = $2^n$ (either in set or not)
\end{itemize}

\textbf{Lecture 24 (4/2): Binomial Coefficients}
\begin{itemize}
    \item binomial coefficients are the coefficients of the binomial expansion of $(x+y)^n$
\end{itemize}

\textbf{Lecture 24 (4/2): Binomial Theorem}
\begin{itemize}
    \item $$(x+y)^n = \sum_{k=0}^n {n \choose k}x^k y^{n-k}$$
    \item Proof: COME BACK TO
\end{itemize}

\textbf{Lecture 24 (4/2): Binomial Coefficient Identities}
\begin{itemize}
    \item ${n \choose k} = {n \choose n-k}$
    \item $2^n = \sum_{k=0}^n {n \choose k}$
    \item ${n \choose k} = {n-1 \choose k}+{n-1 \choose k-1}$
    \item Proof for above: number of ways to take k elements not including 1 fixed element plus number of ways to take k-1 elements from n-1 because already took 1
\end{itemize}

\textbf{Lecture 24 (4/2): Pascal's Triangle}
\begin{itemize}
    \item can be represented as binomial coefficients are as integers
    \item CHECK DIAGRAM FOR DRAWING
\end{itemize}

\textbf{Lecture 24 (4/2): Count Sizes of Unions}
\begin{itemize}
    \item size 2: $|X \cup Y| = |X| + |Y| - |Y\cap X|$
    \item size 3: $|X_1 \cup X_2 \cup X_3| = |X_1|+|X_2|+|X_3|-|X_1 \cap X_2|-|X_2 \cap X_3|-|X_1 \cap X_3|+|X_1 \cap X_2 \cap X_3|$
\end{itemize}

\textbf{Lecture 25 (4/4): Inclusion/Exclusion}
\begin{itemize}
    \item $$|X_1 \cup ... \cup X_n| = \sum|X_i|-\sum_{i<j}|X_i \cap X_j|+\sum_{i<j<k}|X_i \cap X_j \cap X_k|+...+(-1)^{n-1} |X_1 \cap ... \cap X_n|$$
    \item Proof: COME BACK TO
\end{itemize}

\textbf{Lecture 25 (4/4): Derangements}
\begin{itemize}
    \item definition: derangements are permutations with no fixed points (permutation of the elements of a set such that no element appears in its original position)
    \item let $F_i$ be the set of bijections from $X \rightarrow X$ that fix element $i \in X$
    \item let $D_n$ represent the set of derangements of a string of length n: know that the derangements of a string are all the permutations excluding any permutation that has one or more fixed elements:
    $$|D_n| = n! - |F_1 \cup F_2 \cup ... \cup F_n|$$
    \item to count derangements look at PAPER ATTACHED
\end{itemize}

\textbf{Lecture 26 (4/6): Pigeonhole Principle}
\begin{itemize}
    \item if place $m$ objects into $k$ groups, then some group has at least %$\ceil[\big]{\frac{m}{k}}$
    $\ceil{\frac{m}{k}}$ objects
    \item define set A and B, if $|A| > |B|$, then for every total function $f: A \rightarrow B$, there exist 2 different elements of A that are mapped by f to the same element of B
    \item Generalized Pigeonhole Principle: If $|A| > k * |B|$, then every total function $f: A \rightarrow B$ maps at least $k+1$ different elements of A to the same element of B
\end{itemize}

\textbf{Lecture 26 (4/6): Strong Pigeonhole Principle}
\begin{itemize}
    \item Let $m_1, ..., m_n$ be positive integers
    \item if we place $m_1 +...+m_n -n +1$ objects into n boxes, then either box 1 contains at least $m_1$ objects or box 2 contains at least $m_2$ objects, ..., or box n contains at least $m_n$ objects
    \item Proof: COME BACK TO
\end{itemize}
\end{document}

