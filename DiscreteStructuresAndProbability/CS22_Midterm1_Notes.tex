\documentclass[twocolumn]{article}
\usepackage[letterpaper, margin=0.5in]{geometry}
\usepackage{amsmath}
\usepackage{amssymb}

\begin{document}
\textbf{\underline{CS 22: Midterm 1 Study Sheet}} \\

\textbf{Lecture 2 (1/26): Definitions}
\begin{itemize}
    \item proposition: a statement that is either true or false (i.e. $2+3 = 5$)
    \item predicate: a proposition whose truth value depends on a variable (i.e. n is a perfect square)
\end{itemize}

\textbf{Lecture 2 (1/26): Proof by Contradiction}
\begin{itemize}
    \item In order to prove a proposition P by contradiction
    \item 1. state "We use proof by contradiction."
    \item 2. write "suppose P is false"
    \item 3. deduce something know to be false (a logical contradiction)
    \item 4. write "This is a contradiction, therefore P must be true"
    \item alternate form: If p then q; assume that q is not true and show that if q is not true then p is not true (contrapositive)
\end{itemize}

\textbf{Lecture 3 (1/29): Number Systems}
\begin{itemize}
    \item $\mathbb{Z}$ Integers
    \item $\mathbb{Q}$ Rationals: any number that can be expressed as a fraction of two integers
    \item $\mathbb{R}$ Reals
\end{itemize}

\textbf{Lecture 4 (1/31): Set Equality}
\begin{itemize}
    \item Set equality: For two sets A and B, A=B if and only if $A \subseteq B$ and $B \subseteq A$
    \item Proving set equality by subset containment in this way is called the set element method
\end{itemize}

\textbf{Lecture 5 (2/2): Cartesian Product}
\begin{itemize}
    \item Let A, B be sets. Define $A \times B = \{ (a,b) | a \in A \text{ and } b \in B\}$ where $(a, b)$ is an ordered pair.
\end{itemize}

\textbf{Lecture 5 (2/2): Relations}
\begin{itemize}
    \item A relation R on $A \times B$ is any subset of $A \times B$, $R \subseteq A \times B$ 
    \item A relation $R \subseteq A \times A$ is called a "relation on A"
    \item ways to denote a pair (a, b) is in relation R: $(a, b) \in R$ or $aRb$
\end{itemize}

\textbf{Lecture 5 (2/2): Properties of Relations}
\begin{itemize}
    \item Let R be a relation on A
    \begin{itemize}
        \item Reflexive: R is called Reflexive if $\forall a \in A, (a, a) \in R$
        \item Symmetric: R is called symmetric if $\forall a, b \in A, (a, b) \in R \implies (b, a) \in R$
        \item Transitive: R is called transitive $\forall a, b, c \in A$ if $(a, b) \in R$ and $(b, c) \in R \implies (a,c) \in R$
        \item Equivalence Relation: a relation that is reflexive, symmetric and transitive.
        \item Partial Order: a relation R is a partial order on a set S if and only if it is reflexive, antisymmetric, and transitive.
        \item Antisymmetry: if $(x, y) \in R$ and $(y, x) \in R$, then $x=y$.
    \end{itemize}
\end{itemize}

\textbf{Lecture 6 (2/5): Equivalence Classes}
\begin{itemize}
    \item Define R to be an equivalence relation on A $(R \subseteq A \times A)$, then $[a]_R = \{ x \in A | (x, a) \in R\}$
    \item $[a]_R$ is the equivalence class of a
    \item because R is an equivalence relation and by symmetry, it is implied that $[a]_R = \{ x \in A | (a, x) \in R\}$
\end{itemize}

\textbf{Lecture 6 (2/5): Partitions}
\begin{itemize}
    \item A partition of a set A is a collection of disjoint subsets of A that cover A. So $P = B_1B_2...B_n$ where $B_i\subseteq A$
    \begin{itemize}
        \item $\forall a \in A, a \in B$ for some i
        \item $\forall i, jB_i \cap B_j = \emptyset$
    \end{itemize}
\end{itemize}

\textbf{Lecture 7 (2/9): Functions}
\begin{itemize}
    \item given $f: A \rightarrow B$, A is known as the domain of the function, B is known co-domain of the function
    \item image/range = $
    \{ b \in B | \exists a \in A \text{ s.t. } f(a) = b\}$
    \item $\forall a \in A \exists$ exactly one pair (a, b) in the relation where $b \in B$ (vertical line test)
\end{itemize}

\textbf{Lecture 7 (2/9): Injectivity, Surjectivity, Bijectivity}
\begin{itemize}
    \item Surjectivity: f is surjective (onto) if $\forall b \in B \exists a \in A \text{ s.t. } f(a) = b$
    \item Injectivity: f is injective (one-to-one) if $\forall b \in B \exists$ at most one $a \in A \text{ s.t. f(a) = b}$
    \item Bijectivity: f is bijectivte if it is both injective and surjective
    \item Define f to be a bijective function $f: A \rightarrow B$
    \item f is injective $\implies$ $|B| \geq |A|$
    \item f is surjective $\implies$ $|B| \leq |A|$
    \item f is bijective $\implies$ $|B| = |A|$
\end{itemize}

\textbf{Lecture 9 (2/14): Bijections between infinite sets}
\begin{itemize}
    \item Two sets A, B have the same cardinality if $\exists$ a bijection $f: A \rightarrow B$.
\end{itemize}

\textbf{Lecture 10 (2/16): Induction}
\begin{itemize}
    \item prove propositions p(n) are true $\forall n \geq b$ $b, n \in \mathbb{Z}$
    \item base case: $p(b)$
    \item inductive step: $p(k) \implies p(k+1)$
    \item ordinary induction:
    \item 1. state that the proof uses induction
    \item 2. define an appropriate predicate
    \item 3. prove that P(0), the base case, is true
    \item 4. prove that P(n) implies P(n+1) for every non-negative integer n
    \item 5. invoke induction
\end{itemize}

\textbf{Lecture 11:(2/21)}
\begin{itemize}
    \item Strong induction: Let P be a predicate on non-negative integers
    \item if P(0) is true and $\forall n \in \mathbb{N}$, P(0), P(1), ...,P(n) together imply P(n+1), then P(m) is true for all $m \in \mathbb{N}$
\end{itemize}
\end{document}  